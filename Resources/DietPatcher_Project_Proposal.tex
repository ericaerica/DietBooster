\documentclass{article}
\usepackage{indentfirst}
\usepackage{pgfgantt}

\begin{document}

\title{\huge \textbf{DietPatcher}}
\author{Erica Tomaselli \& Stanko Franz Ovkaric}
\date{}
\maketitle


\section{Introduction}


Generally, people are not really aware about what they are/should be/should not be eating. Recently, however, many have started seeking information about food and diets; to this end, the Internet is a great and mostly used resource, thus many websites already exist which provide nutritional information to the users at various levels of personalization: some have a passive, dictionary-like approach based on lists or tables, while others show more individualized results based on user input.

Nonetheless the most personalized websites tell the users what nutrients they are/are not assuming and in which percentage, but not where to get the missing ones, so the problem is diagnosed but not solved.

\section{Our solution}
Our web-application offers a personalized solution to this issue. The user is able to track his/her own diet and receive suggestions on how to improve it based on his/her own profile information, food habits and tags.

\section{Features}
\begin{small}
	\textasteriskcentered\textasteriskcentered\textasteriskcentered \space Legend \space \textasteriskcentered\textasteriskcentered\textasteriskcentered\\
	\begin{itemize}
		\renewcommand\labelitemi{{\boldmath$\cdot$}}
		\item Nutrient: nutritional component in foods; this category includes all macronutrients (proteins, carbohydrates and fats), micronutrients (vitamins, minerals and organic acids) and fibers.
		\item Food: any substance made of nutrients that can be consumed by the user for nutritional purposes (e.g. any fruit, vegetable or meat without additional condiments); this category also includes drinks, common composite foods (e.g. pasta, tomato sauce, French fries) and foods that are not usually consumed alone (e.g. flour, oil).
		\item Meal: combination of foods and/or dishes that the user plans to eat during the selected day.
		\item Meal plan: set of meals the user plans to eat on the selected day.\\
	\end{itemize}
	\hspace*{0.5cm}\textasteriskcentered\textasteriskcentered\textasteriskcentered\textasteriskcentered\textasteriskcentered\textasteriskcentered\textasteriskcentered\textasteriskcentered\textasteriskcentered\textasteriskcentered\textasteriskcentered\textasteriskcentered\textasteriskcentered\textasteriskcentered\textasteriskcentered\\
\end{small}

The user has a username and a password to access his/her own account. Every account has a profile page and a meal plan page.\\
\\
In the profile page the user is able to:\\

\begin{itemize}

  \item insert personal information: name, age, gender, height, weight, waist size.
  \item have tags to identify in a user category, like "Vegetarian", "Vegan", "Paleo" or "Frutarian".
  \item visualize his/her BMI (Body Mass Index) and BAI (Body Adiposity Index), that the application will calculate based on the personal information.\\

\end{itemize}
In the meal plan page the user is able to:\\
\begin{itemize}

  \item Navigate through the calendar to select the meal plan of a certain day.
  \item Select from the DB the foods he/she has eaten or plans to eat, select the amount in grams and add it to the current meal plan.
  \item Check how nutritious a daily plan is by looking at the percentage of covered nutrients. These percentages are shown with colored bars that indicate if it's not enough (yellow), if it's ok (green) or if it's too much (red).
  \item See a real-time modification of the percentages based on the foods added to the meal plan as soon as the user adds them.
  \item See recommendations based on what foods has the user eaten in the past, what other people with the same tags have eaten and in general what is one of the most suitable food to improve his/her own diet. The user can select the nutrient he/she wants recommendations for, taken from a list of "lacking" nutrients.
\end{itemize}


\section{Technologies}

\begin{itemize}

  \item Java 8
  \item Apache Tomcat
  \item Java Servlets
  \item JSP
  \item CSS3
  \item Javascript, jQuery
  \item HTML
  \item PostgreSQL
\end{itemize}
\clearpage

\section{Database Structure}
\subsection{Food and nutritional values}
In this project we are using the USDA National Nutrient Database for Standard Reference (http://www.ars.usda.gov/Services/docs.htm?docid=8964) which contains four principal and eight support tables offering the nutritional values of 8618 different foods.
The four principal tables are:
- Food Description
- Nutrient Data
- Weight



\section{Time estimation}
Weeks division:\\ 
1st week: 23 Mar / 29 Mar	\textperiodcentered 
2nd week: 30 Mar / 5 Apr	\textperiodcentered
3rd week: 6 Apr / 12 Apr	\textperiodcentered
4th week: 13 Apr / 19 Apr	\textperiodcentered
5th week: 20 Apr / 26 Apr	\textperiodcentered
6th week: 27 Apr / 3 May	\textperiodcentered
7th week: 4 May / 10 May	\textperiodcentered
8th week: 11 May / 17 May	\textperiodcentered
9th week: 18 May / 24 May	\textperiodcentered
10th week: 25 May / 31 May.\\

\hspace*{-3cm}
\begin{ganttchart}[vgrid=true]{1}{20}

\gantttitle{Weeks}{20} \\
\gantttitlelist{1,...,10}{2} \\
\ganttbar{Software Proposal}{1}{2} \\
\ganttbar{Gather data sets}{1}{2} \\
\ganttbar{Project creation}{1}{2} \\
\ganttbar{Logo creation}{1}{2} \\
\ganttbar{Login + Registration pages creation}{3}{4} \\
\ganttbar{Database structure}{4}{7} \\
\ganttbar{Population of DB with food data sets}{5}{6} \\
\ganttbar{Profile page creation}{7}{8} \\
\ganttbar{Calendar GUI and management}{9}{11} \\
\ganttbar{Recommender GUI}{12}{13} \\
\ganttbar{Recommender Algorithm}{12}{13} \\
\ganttbar{Nutrients GUI}{14}{15} \\
\ganttbar{Documentation}{16}{17} \\
\end{ganttchart}


\end{document}